\chapter{Problema de Investigación}
Es la identificación de la problemática que se trata de solucionar por medio de la investigación y, para la tesis, es en sí la elección del tema que servirá de base para elaborarla mediante una proposición concreta indicando el contexto preliminar del problema, definición del problema, antecedentes y/o estado del arte preliminar, objetivos, justificación/relevancia.
El problema deberá cumplir una serie de condiciones que de alguna forma justifiquen el esfuerzo necesario para resolverlo. Entre ellas: originalidad, trascendencia, actualidad, relevancia y la posibilidad de permitir el uso de lo aprendido a lo largo de la carrera.

\section{Descripción del problema}
Descripción del contexto y definición del problema. Se describe la situación problemática del contexto desde la perspectiva científica. Debe enunciarse referencias que sustenten la problemática

\section{Objetivos de la Investigación}
Es la definición de lo que se pretende cumplir con la tesis. Debe existir una estricta correspondencia entre los objetivos, el planteamiento del problema y las conclusiones. También demanda una redacción sencilla, concreta y que contemple las siguientes reglas:

\begin{itemize}
    \item Iniciar el objetivo con un verbo en infinitivo.
    \item Determinar primero el qué se quiere y después el para qué se hace.
    \item Limitar la redacción a frases sustantivas.
\end{itemize}

%% La autoindentación dentro de subsecciones no es posible por lo que estas se manejan como ambientes separados.
\begin{subseccion}{Objetivo General}
Descripción de la finalidad principal que persigue la investigación, el motivo que le dará vigencia. El objetivo general y las preguntas de investigación están íntimamente relacionados entre si, por lo que deben ser coherentes
\end{subseccion}

\begin{subseccion}{Objetivos Específicos}
Señalan las actividades que se deben cumplir para avanzar en la investigación y lo que se pretende lograr en cada una de las etapas de ella, por ende, la suma de los resultados de cada uno de los objetivos específicos permitirá alcanzar el propósito integral del objetivo general. Especifica los logros concatenados que se pretende conseguir.
Para la formulación de los objetivos considere lo siguiente:
\begin{itemize}
    \item Deben estar dirigidos a los elementos básicos del problema
    \item Deben ser medibles y observables
    \item Deben ser claros y precisos
    \item Su formulación debe involucrar resultados concretos
    \item El alcance de los objetivos debe estar dentro de las posibilidades del investigador
    \item Deben ser expresados en verbos en infinitivos
\end{itemize}
\end{subseccion}

\section{Preguntas de Investigación}
Una vez que se tiene bien claro el problema, se redactan las preguntas de investigación de acuerdo al problema que se analizará. La pregunta de investigación es uno de los primeros pasos metodológicos que un investigador debe llevar a cabo cuando emprende una investigación. La pregunta de investigación debe ser formulada de manera precisa y clara, de tal manera que no exista ambigüedad respecto al tipo de respuesta esperado.
Las preguntas de investigación son operaciones mentales que hace el investigador para reconocer los puntos que le interesa abordar en su investigación. Por lo tanto, cuando las preguntas están planteadas incorrectamente el razonamiento lógico no entiende cuál es la operación que debe realizar. Las preguntas de investigación deben contener las siguientes características:
\begin{enumerate}
    \item Ser concretas: es decir no dar cabida a la vaguedad. Vaguedad significa que no se entiende exactamente por qué cosa pregunta.
    \item Ser claras: es decir dejar evidente lo que se pregunta.
    \item Ser precisas: es decir puntuales y exactas en lo que preguntan.
    \item Estar completas, es decir sobre todo que contengan sujeto o predicado.
    \item Siempre deben contener un adverbio de pregunta
\end{enumerate}

La pregunta de investigación puede ser una afirmación o un interrogante acerca del fenómeno, en forma precisa y clara, de tal forma que de ésta se desprendan los métodos, procedimientos e instrumentos.

Considere que No todas las investigaciones tienen hipótesis; todo depende del grado de conocimiento sobre el problema que se investiga. Sólo necesitan hipótesis las investigaciones que ya han rebasado la fase exploratoria y se encuentran en fase confirmatoria o verificatoria. Las hipótesis, son justamente el objeto de la confirmación o verificación. Intentar forzar la presencia de hipótesis cuando el conocimiento sobre un problema o la propia naturaleza de dicho problema no lo consienten es uno de los errores más frecuentes que se comenten en la práctica.
